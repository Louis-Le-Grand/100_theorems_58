\documentclass{article}
\usepackage[english]{babel}
\usepackage{../lib/tex/naproche} %<-- falls du lib in deinem Arbeitsverzeichnis hast
%\usepackage{./lib/tex/naproche} falls du lib nicht in deinem Arbeitsverzeichnis hast
\usepackage{amssymb}
\usepackage{hyperref}

\begin{document}

\newcommand{\Prod}[3]{#1_{#2} \cdots #1_{#3}}
\newcommand{\Seq}[2]{\{#1,\dots,#2\}}
\newcommand{\FinSet}[3]{\{#1_{#2},\dots,#1_{#3}\}}
\newcommand{\Primes}{\mathbb{P}}
\newcommand{\pow}{{\cal P}}
\newcommand{\range}{\operatorname{ran}}
\newcommand{\inv}[1]{#1^{-1}}
\newcommand{\sset}[2]{\{#1\}_{#2}}

\title{Formalizing the binomial coefficient in \Naproche{}}

\author{Johannes Folltmann, Simon Korswird,\\
Ludwig Monnerjahn, Lars Nießen, Hannah Scholz\\
\emph{Mathematical Institute, University of Bonn, Germany}
}

\maketitle

\section{Introduction}
In this document we give a proof on the binomial coefficient and its relation to the $k$-element subsets of an $n$-element set.
This is part of the \href{https://www.cs.ru.nl/~freek/100/}{100 theorems challenge} which is being tracked by Freek Wiedijk. Here we will prove the "Formula for the Number of Combinations"
(\#58). 

To give an overview of this document we explain the purpose of the different sections. 
The first sections deals with some preliminaries. We mainly use this section to prove that the subset of a finite set is finite.
Then we introduce the factorial in the third section.
We prove some basic facts about the factorial which we will use in the later sections. The fourth section introduces 
the binomial coefficient and its basic properties. The most important proof of this section is the sum of binomial coefficients $\binom{n}{k} + \binom{n}{k-1} = \binom{n+1}{k}$.

We prove the actual theorem (\# 58) in section 5. To make it easier to process for the \Naproche system we introduce some new signutures. 
That way we are able to make the statement as compact as possible. The proof uses induction on the size of the set
and in the induction step we use the sum of binomial coefficients as mentioned above. To use this identity we split the set of k-element subsets
into two disjoint sets where one is the set of subsets which contain a fixed element and the other is the set of subsets which don't. 

To use some facts about sets and numbers we load the file \textit{100\_theorems.ftl.tex}. We will prove any additionally needed statement on our way.
\begin{forthel}
   [prove off][check off]
   [readtex \path{100_theorems.ftl.tex}]
   [prove on][check on]
\end{forthel}

\section{Preliminaries}

\begin{forthel}
	
\begin{lemma}[reduction of fractions]
	Let $k,l,n$ be integers such that $l,n$ are nonzero.
	Then $\frac{k}{l} = \frac{k\cdot n}{l \cdot n}$. 
\end{lemma}

\begin{lemma}
	Let $l$,$m$ and $n$ be integers such that $n$ is nonzero.
	Then $\frac{l}{n} + \frac{m}{n} = \frac{l + m}{n}$.
\end{lemma}

\begin{lemma}
	For all natural numbers $n$
	if $x \in \Seq{1}{n}$
	then $\sset{x}{\Seq{1}{n}}$ is finite and $|\sset{x}{\Seq{1}{n}}| = 1$. 
\end{lemma}
\begin{proof}[by induction on $n$]
	Let $n$ be a natural number.
	
	Case $n=0$. Trivial.
	
	Case $n > 0$. 
	Let $x \in \Seq{1}{n}$. 
	
	Case $x = n$. 
	$\sset{n}{\Seq{1}{n}} = \Seq{n}{n}$.
	Define $h(x) = 1$ for $x \in \Seq{n}{n}$. 
	$h$ is a bijection between $\Seq{n}{n}$ and $\Seq{1}{1}$. 
	Qed.
	
	Case $x \neq n$.
	$x < n$.
	Let $m$ be natural number such that $m = n-1$.
	$m$ is inductively smaller than $n$.
	Then $x \leq m$ and $x \in \Seq{1}{m}$.
	Then $\sset{x}{\Seq{1}{m}}$ is finite and $|\sset{x}{\Seq{1}{m}}|=1$.
	$\sset{x}{\Seq{1}{m}}=\sset{x}{\Seq{1}{n}}$.
	Qed.
	
	Qed.
\end{proof}

\begin{lemma}
	Let $A$ be a set. Let $a\in A$. Then $\sset{a}{A}$ is finite and $| \sset{a}{A} | = 1$.
\end{lemma}
\begin{proof}
	Define $f(x) = 1$ for $x \in \sset{a}{A}$.
	$f$ is a map from $\sset{a}{A}$ to $\Seq{1}{1}$.
	$f$ is a surjection from $\sset{a}{A}$ onto $\Seq{1}{1}$.
	$f$ is injective.
	Thus $f$ is a bijection between $\sset{a}{A}$ and $\Seq{1}{1}$.
\end{proof}

\end{forthel}
%
We now prove that every subset of a finite set is finite.
%
\begin{forthel}

\begin{lemma}
	For all sets $A$ and all natural numbers $n$ if $A \subseteq \Seq{1}{n}$ then $A$ is finite.
\end{lemma}

\begin{proof}[by induction on $n$]
	Let $A$ be a set and $n$ be a natural number such that $A \subseteq \Seq{1}{n}$.
	
	Case $n=0$. Then $\Seq{1}{0} = \emptyset$. Qed.
	
	Case $n \neq 0$. Take a natural number $m$ such that $m+1=n$. 
	Then $m$ is inductively smaller than $n$. 
	
	$\Seq{1}{n} = \Seq{1}{m} \cup (\Seq{1}{n} \setminus \Seq{1}{m})$.\\
	$A = (A \cap \Seq{1}{m}) \cup (A\cap (\Seq{1}{n} \setminus \Seq{1}{m}))$.\\
	$(A \cap \Seq{1}{m}) \subseteq \Seq{1}{m}$.\\
	$A \cap \Seq{1}{m}$ is finite.
	
	Let us show that $A\cap (\Seq{1}{n} \setminus \Seq{1}{m})$ is finite.
	
	$\Seq{1}{n} \setminus \Seq{1}{m} = \Seq{1}{m+1} \setminus \Seq{1}{m}$.\\
	Thus $x \in \Seq{1}{m+1} \setminus \Seq{1}{m}$ for every $x \in A\cap (\Seq{1}{n} \setminus \Seq{1}{m})$.\\
	Therefore $x = m+1$ for every $x \in A\cap (\Seq{1}{n} \setminus \Seq{1}{m})$.\\
	Thus $A\cap (\Seq{1}{n} \setminus \Seq{1}{m}) \subseteq \sset{m+1}{\Seq{1}{m+1}}$.
	
	Case $A\cap (\Seq{1}{n} \setminus \Seq{1}{m}) = \emptyset$. Trivial.
	
	Case not $A\cap (\Seq{1}{n} \setminus \Seq{1}{m}) = \emptyset$.
	Then $A\cap (\Seq{1}{n} \setminus \Seq{1}{m}) = \sset{m+1}{\Seq{1}{m+1}}$.
	$\sset{m+1}{\Seq{1}{m+1}}$ is finite.
	Qed.
	
	Qed.
	
	$A \cap \Seq{1}{m}$ and $A\cap (\Seq{1}{n} \setminus \Seq{1}{m})$ are disjoint.
	Therefore $A$ is finite.
	Qed.
\end{proof}

\begin{lemma}
	Let $A$ be a set and $B$ be a finite set such that $A \subseteq B$.
	$A$ is finite.
\end{lemma}
\begin{proof}
	Take a natural number $n$ such that $B$ and $\Seq{1}{n}$ are equinumerous.
	Take a bijection $f$ between $B$ and $\Seq{1}{n}$.
	Then $f[A]$ is a subset of $\Seq{1}{n}$.
	$f[A]$ is finite.
	Define $g(x) = f(x)$ for $x\in A$.
	$g$ is a map from $A$ to $f[A]$.
	$g$ is injective.
	$g$ is a surjection from $A$ onto $f[A]$.
	Thus $g$ is a bijection between $A$ and $f[A]$.
	Therefore $A$ and $f[A]$ are equinumerous and $A$ is finite.
\end{proof}

\begin{lemma}
	For all nonzero natural numbers $n$ and all finite sets $A, B$
	if $B \subseteq A$ and $|A| = n$ and $A \neq B$
	then $|B| < |A|$.
\end{lemma}
\begin{proof}
	Let $n$ be nonzero natural number.
	Let $A,B$ be finite sets such that 
	$B \subseteq A$ and $|A|=n$ and $A \neq B$.
	If $A \neq B$ and $B \subseteq A$ then
	$A \setminus B$ is nonempty.
	$A \setminus B \subseteq A$.
	$A \setminus B$ is finite.
	$A \setminus B$ and $B$ are disjoint.
	$A = (A \setminus B) \cup B$.
	Then $|A| = |(A \setminus B)| + |B|$.
\end{proof}


\end{forthel}

\section{Factorial}

\begin{forthel}

Let $n$ denote a natural number.
\begin{signature}
$n!$ is a natural number.
\end{signature}

\begin{axiom}
$0!=1$.
\end{axiom}

\begin{axiom}
$(n+1)!=(n!) \cdot (n+1)$.
\end{axiom}

\begin{lemma}
Let $k,n$ be nonzero natural numbers. $k \cdot n \neq 0$. 
\end{lemma}

\begin{theorem}
For every natural number $k$ $k! \neq 0$.
\end{theorem}
\begin{proof}[by induction on $k$]
Let $k$ be a natural number.

Case $k = 0$.Then $k! = 1$. Thus $k! \neq 0$. Qed.

Case $k = 1$. Then $k! = 1 \neq 0$. Qed.

Case $k > 1$. $k = (k-1)+1$. $(k-1)! \neq 0$. $k \neq 0$. Then $k, (k-1)!$ are nonzero natural numbers.
Then $(k \cdot (k-1)!) \neq 0$. Thus $k! \neq 0$.  Qed.
\end{proof} 

\end{forthel}

\section{Binomial coefficients}

\begin{forthel}

Let $n, k$ denote natural numbers.
\begin{signature}
$\binom{n}{k}$ is a natural number.
\end{signature}


\begin{axiom}
Let $k > n$. $\binom{n}{k} = 0$.
\end{axiom}
    
\begin{axiom}[definition binomial coefficient]
Let $k \leq n$. $\binom{n}{k} = \frac{n!}{k! \cdot (n-k)!}$.
\end{axiom}

\begin{lemma}
$\binom{0}{0} = 1$.
\end{lemma}
\begin{proof}
$0! = 1$.
$0$ is a natural number and $0 \leq 0$.
Take natural numbers $n,k$ such that $n=k=0$ and $k\leq n$.
$\binom{0}{0} = \binom{n}{k} = \frac{n!}{k! \cdot (n-k)!} = 
\frac{0!}{0! \cdot (0-0)!} = \frac{1}{1 \cdot 1} = 1$
(by definition binomial coefficient).
\end{proof}


\begin{theorem}
Let $k,n$ be natural numbers such that $1 < k < n$.
$\binom{n}{k} > 0$. 
\end{theorem}
\begin{proof}
Let $n,k$ be nonzero natural numbers such that $1 < k < n$.
Then $n! > 0$ and $k! > 0$ and $(n-k)! > 0$. 
Then $\frac{n!}{k! \cdot (n-k)!} > 0$. 
Then $\binom{n}{k} > 0$. 
\end{proof}

\begin{lemma}
Let $n,k$ be nonzero natural numbers such that $1 < k < n$.
Then $n-(k-1)$ is a natural number.
\end{lemma}
\begin{proof}
$k-1$ is a natural number. $k-1 < k < n$. $k-1 \leq n$.
\end{proof}
  
\begin{lemma}
Let $n,k$ be nonzero natural numbers such that $1 < k < n$.
$n-(k-1) = (n-k)+1$.
\end{lemma}
  
\begin{theorem}[Binomial Sum]
Let $n,k$ be nonzero natural numbers such that $1 < k \leq n$.
Then $\binom{n}{k} + \binom{n}{k-1} = \binom{n+1}{k}$.
\end{theorem}
\begin{proof}
	
Let us show that
$\binom{n}{k} + \binom{n}{k-1} = \frac{n!}{k! \cdot (n-k)!} + \frac{n!}{(k-1)! \cdot (n-(k-1))!}$.

$k \leq n$.
$k-1 \leq k \leq n$.
$k-1 \leq n$.

Indeed we can show that $k-1 \leq n$.

Case $k = n$. Trivial.

Case $k < n$. Trivial.

End.

$\binom{n}{k} = \frac{n!}{k! \cdot (n-k)!}$.

$\binom{n}{k-1} = \frac{n!}{(k-1)! \cdot (n-(k-1))!}$.

Qed.
\\
 
Let us show that \\
$\frac{n!}{k! \cdot (n-k)!} + \frac{n!}{(k-1)! \cdot (n-(k-1))!} = 
\frac{n! \cdot ((n-k)+1)}{(k! \cdot (n-k)!) \cdot ((n-k)+1) } + 
\frac{n! \cdot k}{((k-1)! \cdot ((n-k)+1)!) \cdot k}$.

$n!$, $k! \cdot (n-k)!$ and $(n-k)+1$ are natural numbers.
$k! \cdot (n-k)!$ and $(n-k)+1$ are nonzero.

Take a nonzero natural number $l$ such that $l = k! \cdot (n-k)!$.
Take a nonzero natural number $m$ such that $m = (n-k)+1$.

Then $\frac{n!}{(k! \cdot (n-k)!)} = \frac{n!}{l} = \frac{n! \cdot m}{l \cdot m} = 
\frac{n! \cdot ((n-k)+1)}{l \cdot ((n-k)+1) } =  \frac{n! \cdot ((n-k)+1)}{(k! \cdot (n-k)!) \cdot ((n-k)+1) }$.

$n-(k-1) = (n-k) +1$.
$n!$, $(k-1)! \cdot (n-(k-1))!$ and $k$ are natural numbers.
$(k-1)! \cdot (n-(k-1))!$ and $k$ are nonzero.

Take a nonzero natural number $o$ such that $o = (k-1)! \cdot (n-(k-1))!$.
Then $o = (k-1)! \cdot ((n-k)+1)!$.

[timelimit 10]
Therefore $\frac{n!}{(k-1)! \cdot (n-(k-1))!} = \frac{n!}{o} = \frac{n! \cdot k}{o \cdot k}
= \frac{n! \cdot k}{((k-1)! \cdot ((n-k)+1)!) \cdot k}$ (by reduction of fractions).
[timelimit 3]

Qed.
\\
  
Let us show that 

$\frac{n! \cdot ((n-k)+1)}{(k! \cdot (n-k)!) \cdot ((n-k)+1) } + 
\frac{n! \cdot k}{((k-1)! \cdot ((n-k)+1)!) \cdot k} = 
\frac{n! \cdot ((n-k)+1)}{k! \cdot ((n-k)+1)! } + 
\frac{n! \cdot k}{k! \cdot ((n-k)+1)!}$.

$(n-k)! \cdot ((n-k)+1) = ((n-k)+1)!$. \\
$(k! \cdot (n-k)!) \cdot ((n-k)+1) = k! \cdot ((n-k)+1)!$. \\
$(k-1) + 1 = k$.
$(k-1)! \cdot k = k!$.

Take natural numbers $l,m$ such that $l = (k-1)!$ and $m = ((n-k)+1)!$.
$((k-1)! \cdot ((n-k)+1)!) \cdot k = 
(l \cdot m) \cdot k =
(l \cdot k) \cdot m =
((k-1)! \cdot k) \cdot ((n-k)+1)! = k! \cdot ((n-k)+1)!$.

Qed.
\\
  
Let us show that 
$\frac{n! \cdot ((n-k)+1)}{k! \cdot ((n-k)+1)! } + 
\frac{n! \cdot k}{k! \cdot ((n-k)+1)!} = 
\frac{(n! \cdot ((n-k)+1)) + (n! \cdot k)}{k! \cdot ((n-k)+1)!}$.

$n! \cdot ((n-k)+1)$, $n! \cdot k$ and $k! \cdot ((n-k)+1)!$ are natural numbers.
$k! \cdot ((n-k)+1)!$ is nonzero.

Qed.
\\

Let us show that
$ \frac{(n! \cdot ((n-k)+1)) + (n! \cdot k)}{k! \cdot ((n-k)+1)!}
= \frac{(n+1)!}{k! \cdot ((n-k)+1)!}$.

$(((n-k)+1)+k) = n+1$.
$(n! \cdot ((n-k)+1)) + (n! \cdot k) = n! \cdot (((n-k)+1)+k) =  n! \cdot (n+1) = (n+1)!$.

Qed.
\\

$k! \cdot ((n+1)+(-k))!$ is nonzero.

Let us show that $ \frac{(n+1)!}{k! \cdot ((n-k)+1)!}
= \frac{(n+1)!}{k! \cdot ((n+1)+(-k))!}$.

$((n-k)+1)=(n+1)+(-k)$.

Qed.
\\

[timelimit 11]
Let us show that $\frac{(n+1)!}{k! \cdot ((n+1)+(-k))!} = \binom{n+1}{k}$.

$k \leq (n+1)$.

Qed.
[timelimit 3]\\

$(n+1)+(-k)$ is a natural number.

Therefore 
$\binom{n}{k} + \binom{n}{k-1}
= \frac{n!}{k! \cdot (n-k)!} + \frac{n!}{(k-1)! \cdot (n-(k-1))!}
= \frac{n! \cdot ((n-k)+1)}{(k! \cdot (n-k)!) \cdot ((n-k)+1) } + 
\frac{n! \cdot k}{((k-1)! \cdot ((n-k)+1)!) \cdot k}
= \frac{n! \cdot ((n-k)+1)}{k! \cdot ((n-k)+1)! } + 
\frac{n! \cdot k}{k! \cdot ((n-k)+1)!}
= \frac{(n! \cdot ((n-k)+1)) + (n! \cdot k)}{k! \cdot ((n-k)+1)!}
= \frac{(n+1)!}{k! \cdot ((n-k)+1)!}
= \frac{(n+1)!}{k! \cdot ((n+1)+(-k))!}
= \binom{n+1}{k}$.
\end{proof}

\begin{lemma}
$\binom{n}{n} = 1$ for all nonzero natural numbers $n$.
\end{lemma}
\begin{proof}
Let $n$ be a nonzero natural number.
$n \leq n.$
$n! \cdot (n-n)! \neq 0$.
$\binom{n}{n} = \frac{n!}{n! \cdot (n-n)!}$.
$n! \cdot (n-n)! = n!$.
$\frac{n!}{n! \cdot (n-n)!} = \frac{n!}{n!} = 1$.
\end{proof}


\begin{lemma}
Let $n,k$ be nonzero natural numbers such that $1 < k \leq n$.
Then $\binom{n-1}{k} + \binom{n-1}{k-1} = \binom{n}{k}$.
\end{lemma}
\begin{proof}
Case $k = n$.
Then $1 < k \leq n$.
Therefore $\binom{n}{k} + \binom{n}{k-1} = \binom{n+1}{k}$.
Qed.
Case $k = 1$. Trivial.
Case $1 < k < n$.
Let $m,l$ be nonzero natural numbers 
such that $1 < l \leq m$ and $m+1 =n$ and $l = k$.
Then $\binom{m}{l} + \binom{m}{l-1} = \binom{m+1}{l}$.
Qed.
\end{proof}

\begin{lemma}
$\binom{n}{0}=1$ for all natural numbers $n$.
\end{lemma}
\begin{proof}
Let $n$ be a natural number.
$\binom{n}{0} 
= \frac{n!}{0! \cdot (n-0)!} 
= \frac{1 \cdot n!}{1 \cdot n!} 
= 1$.
\end{proof}

\begin{lemma}
$\binom{n}{1}=n$ for all nonzero natural numbers $n$.
\end{lemma}
\begin{proof}
Let $n$ be a nonzero natural number.
$1! \cdot (n-1)! \neq 0$.
$1 \leq n$.
$\binom{n}{1} = \frac{n!}{1! \cdot (n-1)!}$.
$(n-1)! \neq 0$.
$((n-1)+1)! = (n-1)! \cdot ((n-1)+1)$.
$(n-1)+1 = n$.
$\frac{n!}{1! \cdot (n-1)!} 
= \frac{n \cdot (n-1)!}{1 \cdot (n-1)!}$.
$\frac{n \cdot (n-1)!}{1 \cdot (n-1)!}
= \frac{n}{1}$.
\end{proof}

\end{forthel}

\section{Formula for the Number of Combinations (\#58)}

We firstly introduce some notation to shorten the statements.
\begin{forthel}

\begin{signature}
$P_{n,k}$ is a set.
\end{signature}

\begin{axiom}[subsets of specific cardinality 1]
Let $n,k$ be natural numbers. \\
$P_{n,k} = \{ x | x \subseteq \Seq{1}{n}$ and $|x|=k \}$.
\end{axiom}

\begin{lemma}
Let $n,k$ be natural numbers such that $k>n$.
$P_{n,k} = \emptyset$.
\end{lemma}
\begin{proof}[by contradiction]
Assume $P_{n,k}$ is nonempty.
Take $x \in P_{n,k}$.
Then $x \subseteq \Seq{1}{n}$.
$|x|=k$.
$|x| \geq |\Seq{1}{n}|$. Contradiction.
\end{proof}

\begin{lemma}
$P_{0,0}$ is finite and $|P_{0,0}|=1$.
\end{lemma}
\begin{proof}
$|\emptyset|=0$.
$\emptyset \subseteq \Seq{1}{0}$.
Thus $\emptyset \in P_{0,0}$.
Define $h(x)=1$ for $x \in P_{0,0}$.
$h$ is bijection between $P_{0,0}$ and $\Seq{1}{1}$.
\end{proof}

\begin{lemma}
Let $n$ be a natural number.
$P_{n,n}$ is finite and $|P_{n,n}|=1$.
\end{lemma}
\begin{proof}
$\Seq{1}{n} \subseteq \Seq{1}{n}$.
$\Seq{1}{n} \in P_{n,n}$.
$P_{n,n} \setminus \sset{\Seq{1}{n}}{P_{n,n}} = \emptyset$.
$P_{n,n} = \sset{\Seq{1}{n}}{P_{n,n}} \cup \emptyset$.
$\sset{\Seq{1}{n}}{P_{n,n}}$ and $\emptyset$ are disjoint.\\
$|\sset{\Seq{1}{n}}{P_{n,n}}|=1$.
\end{proof}

\end{forthel}
%
Now we show the desired theorem for the simpler sets $\Seq{1}{n}$.
%
\begin{forthel}
\begin{theorem}
For all natural numbers $n,k$  $P_{n,k}$ is finite and $|P_{n,k}|= \binom{n}{k}$.
\end{theorem}
\begin{proof}[by Induction on $n$]
Let $n,k$ be natural numbers.

Case $n < k$. Trivial.

Case $n = k$. Trivial.

Case $k < n$.

Case $n=0$.Trivial.

Case $n>0$.

Case $k=0$.
For all $x \in P_{n,k}$ $|x| = 0$.
Therefore $x = \emptyset$ for all $x \in P_{n,k}$.
$\emptyset \subseteq \Seq{1}{n}$. Thus $\emptyset \in P_{n,k}$.
Define $h(x) = 1$ for $x \in P_{n,k}$. 
[timelimit 10]
$h$ is a bijection between $P_{n,k}$ and $\Seq{1}{1}$.
[timelimit 3]
Hence $|P_{n,k}| = 1$ and $\binom{n}{k} = 1$.
Qed.

Case $k>0$.
Let $m$ be a natural number such that $m+1=n$.
$m$ is inductively smaller than $n$.
$|P_{m,k}|=\binom{m}{k}$.
$|P_{m,k-1}|=\binom{m}{k-1}$.
$n \in \Seq{1}{n}$.
$k \leq m$.

Define $P' = \{p| p \in P_{n,k}$ and $n \in p\}$.
$P' \subseteq P_{n,k}$. 


$\Seq{1}{k} \subseteq \Seq{1}{m}$.
$\Seq{1}{k} \in P_{m,k}$.
$P_{m,k}$ is nonempty.

Let us show that $P_{m,k} \subseteq P_{n,k}$.
$|x|=k$ for all $x \in P_{m,k}$.
$x \subseteq \Seq{1}{m}$ for all $x \in P_{m,k}$.
$x \subseteq \Seq{1}{n}$ for all $x \in P_{m,k}$.
$x \in P_{n,k}$ for all $x \in P_{m,k}$(by subsets of specific cardinality 1).
Qed.

$P_{n,k}$ is nonempty.

Let us show that $p \in (P' \cup P_{m,k})$ for all $p \in P_{n,k}$.
Let $p \in P_{n,k}$.

Case $n \in p$.
$p \in P'$. End.

Case $n \notin p$.
$p \subseteq \Seq{1}{m}$.
$\Seq{1}{m} \subseteq \Seq{1}{n}$.
End.

Qed.

$P' \cup P_{m,k} = P_{n,k}$.
$P'$ and $P_{m,k}$ are disjoint.
$P_{n,k} \subseteq \pow(\Seq{1}{n})$.
$P_{n,k}$ is finite.
$P'$ is finite.
$|P_{n,k}|=|P'|+|P_{m,k}|$.

Let us show that $|P'|=|P_{m,k-1}|$.

$P' = \{p| p \in P_{n,k}$ and $n \in p\}$. 
$0 < k \leq n$.

$p \subseteq \Seq{1}{m}$ for every $p \in P_{m,k-1}$.
$p \subseteq \Seq{1}{n}$ for every $p \in P_{m,k-1}$.
$\sset{n}{\Seq{1}{n}} \subseteq \Seq{1}{n}$ for every $p \in P_{m,k-1}$.
$p \cup \sset{n}{\Seq{1}{n}} \subseteq \Seq{1}{n}$ for every $p \in P_{m,k-1}$.
$p \cup \sset{n}{\Seq{1}{n}}$ is a set for every $p \in P_{m,k-1}$.

Define $f(p) = p \cup \sset{n}{\Seq{1}{n}}$ for $p \in P_{m,k-1}$.

Let us show that $f$ is a map from $P_{m,k-1}$ to $P'$.

$p$ and $\sset{n}{\Seq{1}{n}}$ are disjoint for every $p \in P_{m,k-1}$.
$|p| = k-1$ for every $p \in P_{m,k-1}$.
$|\sset{n}{\Seq{1}{n}}| = 1$.
$|f(p)| = |p \cup \sset{n}{\Seq{1}{n}}| = |p| + |\sset{n}{\Seq{1}{n}}| = (k-1)+1  = k$ for every $p \in P_{m,k-1}$.

$p \subseteq \Seq{1}{n}$ for every $p \in P_{m,k-1}$.
$\sset{n}{\Seq{1}{n}} \subseteq \Seq{1}{n}$.
$p \cup \sset{n}{\Seq{1}{n}} \subseteq \Seq{1}{n}$ for every $p \in P_{m,k-1}$.
$f(p) \subseteq \Seq{1}{n}$ for every $p \in P_{m,k-1}$.

$P_{n,k} = \{ x | x \subseteq \Seq{1}{n}$ and $|x|=k \}$.
For all $p \in P_{m,k-1}$  $|f(p)| = k$ and $f(p) \subseteq \Seq{1}{n}$
.
$f(p) \in P_{n,k}$ for every $p \in P_{m,k-1}$.
$n \in f(p)$ for every $p \in P_{m,k-1}$.
$f(p) \in P'$ for every $p \in P_{m,k-1}$.
Qed.

Let us show that $f$ is injective. 

   Let $p,q \in P_{m,k-1}$.
   Assume $f(p) = f(q)$. \\
   We have $f(r) \setminus \sset{n}{\Seq{1}{n}} = (r \cup \sset{n}{\Seq{1}{n}}) \setminus \sset{n}{\Seq{1}{n}} = r$
   for every $r \in  P_{m,k-1}$.
   $p = f(p)\setminus \sset{n}{\Seq{1}{n}} = f(q)\setminus \sset{n}{\Seq{1}{n}} = q$.
   Thus $p=q$.
   Qed.
   
Let us show that $f$ is a surjection from $P_{m,k-1}$ onto $P'$.

   Define $p(q) = q \setminus \sset{n}{\Seq{1}{n}}$ for $q \in P'$.
   $p(q)$ and $\sset{n}{\Seq{1}{n}}$ are disjoint for every $q \in P'$.
   $q \subseteq \Seq{1}{n}$ for every $q \in P'$.
   $q$ is finite for every $q \in P'$.
   $\sset{n}{\Seq{1}{n}}$ is finite.
   $p(q) \subseteq q$ for every $q \in P'$.
   Thus $p(q)$ is finite for every $q \in P'$.
   
   $p(q) \cup \sset{n}{\Seq{1}{n}} = q$ for every $q \in P'$.
   $p(q)$ and $\sset{n}{\Seq{1}{n}}$ are disjoint for every $q\in P'$.
   $|q| = |p(q) \cup \sset{n}{\Seq{1}{n}}|  =  |p(q)| + |\sset{n}{\Seq{1}{n}}|$ for every $q \in P'$.
   Therefore $k = |q| = |p(q)| + |\sset{n}{\Seq{1}{n}}| = |p(q)| + 1$ for every $q \in P'$.
   $k =  |p(q)| + 1$ for every $q \in P'$.
   Thus $|p(q)| = (|p(q)| + 1) -1 =  k-1$ for every $q \in P'$.
   
   $\Seq{n}{n} = \sset{n}{\Seq{1}{n}}$.
   $\Seq{1}{n-1} = \Seq{1}{m}$.\\
   $\Seq{1}{n} = \Seq{1}{n-1} \cup \Seq{n}{n} = \Seq{1}{m} \cup  \sset{n}{\Seq{1}{n}}$.\\
   $\Seq{1}{m} = \Seq{1}{n} \setminus \sset{n}{\Seq{1}{n}}$.
   $p(q) \subseteq \Seq{1}{m}$ for every $q \in P'$.
   For all $q \in P'$ $p(q) \subseteq \Seq{1}{m}$ and $|p(q)| = k-1$.\\
   $P_{m,k-1} = \{x | x \subseteq  \Seq{1}{m}$ and $|x| = k-1 \}$.
   $p(q) \in P_{m,k-1}$ for every $q \in P'$.
   $f(p(q)) = p(q) \cup  \sset{n}{\Seq{1}{n}} = q$ for every $q \in P'$.
   Thus $q \in f[P_{m,k-1}]$ for every $q \in P'$.
   Qed.
   
Therefore $f$ is a bijection between $P_{m,k-1}$ and $P'$.
Thus $P_{m,k-1}$ and $P'$ are equinumerous.

Let us show that $P_{m,k-1}$ is finite.
  $P_{m,k-1} \subseteq \pow(\Seq{1}{m})$.
  $\pow(\Seq{1}{m})$ is finite.
  Qed.
  
$n-1 = m$.
$P_{n-1,k-1} = P_{m,k-1}$.
$P_{m,k-1}$, $P_{n-1,k-1}$ and $P'$ are finite.

For all sets $S,T$ if $S,T$ are finite and equinumerous then $|S|= |T|$.
Therefore $|P_{n-1,k-1}| = |P_{m,k-1}| = |P'|$.
$|P'| = |P_{n-1,k-1}|$.

Qed.

$|P_{m,k}|=\binom{n-1}{k}$.
$|P_{m,k-1}|=\binom{n-1}{k-1}$.
$|P'|=\binom{n-1}{k-1}$.
$\binom{n}{k}=\binom{n-1}{k-1} + \binom{n-1}{k}$.
$|P_{n,k}|=\binom{n}{k}$.

Qed.
Qed.
Qed.
\end{proof}
\end{forthel}
%
We introduce some more notation.
%
\begin{forthel}


\begin{signature}
Let $X$ be a finite set such that $|X| = n$  and $n,k$ be natural numbers.
$P_{X,n,k}$ is a set.
\end{signature} 

\begin{axiom}[subsets of specific cardinality 2]
Let $X$ be a finite set such that $|X| = n$  and $n,k$ be natural numbers.\\
$P_{X,n,k} = \{y|y$ is a finite set such that $y \subseteq X$ and $|y| = k\}$.
\end{axiom}

\begin{lemma}
Let $n,k$ be natural numbers such that $k \leq n$.\\
$P_{\Seq{1}{n},n,k}=P_{n,k}$.
\end{lemma}
\begin{proof}
$P_{n,k} = \{ x | x \subseteq \Seq{1}{n} and |x|=k \}$.
Let $X$ be a finite set such that $|X| = n$.
[timelimit 20]
$P_{X,n,k} = \{y|y$ is a finite set such that $y \subseteq X$ and $|y| = k\}$ (by subsets of specific cardinality 2).
[timelimit 3]
\end{proof}

\begin{lemma}
Let $n,k$ be natural numbers such that $k \leq n$.
$|P_{\Seq{1}{n},n,k}|=|P_{n,k}|$.
\end{lemma}

\begin{lemma}
Let $X$ be finite set such that $|X|=0$.
$\emptyset \in P_{X,0,0}$.
\end{lemma}
\begin{proof}
$\Seq{1}{0} = \emptyset$.
Thus $X = \emptyset$.
$\emptyset \subseteq X$. $|\emptyset|=0$. $\emptyset$ is finite.
$P_{X,0,0} = \{y | y$ is finite set such that $y \subseteq X$ and $|y| = 0\}$ (by subsets of specific cardinality 2).
\end{proof}

\begin{lemma}
Let $n,k$ be natural numbers such that $k>n$. Let $X$ be a finite set such that $|X| = n$.
$P_{X,n,k} = \emptyset$.
\end{lemma}
\begin{proof}[by contradiction]
Assume $P_{X,n,k}$ is nonempty.
Take $x \in P_{X,n,k}$.
$P_{X,n,k} = \{y|y$ is a finite set such that $y \subseteq X$ and $|y| = k\}$ (by subsets of specific cardinality 2).
Then $x \subseteq X$.
$|x|=k$.
$|x| \geq |\Seq{1}{n}|$. Contradiction.
\end{proof}

\end{forthel}
%
We can now move on to show the actual theorem.
%
\begin{forthel}
\begin{lemma}
Let $n,k$ be natural number such that $k \leq n$.
Let $X$ be a finite set such that $|X|=n$.
Then $P_{X,n,k}$ and $P_{n,k}$ are finite and $|P_{X,n,k}|=|P_{n,k}|$.
\end{lemma}
\begin{proof}
	
Case $n=0$.
$k \leq n$.
$k \leq 0$.
$k = 0$.
$\binom{0}{0} = 1$.
$\Seq{1}{0} = \emptyset$.
Thus $X = \emptyset$.
$\emptyset \subseteq X$. $|\emptyset|=k$.

[timelimit 20]
$P_{X,n,k} = \{y | y$ is finite set such that $y \subseteq X$ and $|y| = k\}$ (by subsets of specific cardinality 2).
[timelimit 3]

Thus $\emptyset \in P_{X,n,k}$.
Define $h(x) = 1$ for $x \in P_{X,n,k}$. 
$Y = \emptyset$ for all finite sets $Y$ such that $|Y|=0$.
$|Y|=0$ for all $Y \in P_{X,n,k}$.
$h$ is injective.
$h(\emptyset) = 1$.
$h$ is a surjection from $P_{X,n,k}$ onto $\Seq{1}{1}$.
$h$ is a bijection between $P_{X,n,k}$ and $\Seq{1}{1}$.
Hence $|P_{X,n,k}| = 1$ and $|P_{n,k}| = 1$. End.

Case $n>0$.

Case $k=0$.
$|X| = n$.
$\emptyset \subseteq X$. $|\emptyset|=k$.
$P_{X,n,k} = \{y | y$ is finite set such that $y \subseteq X$ and $|y| = k\}$ (by subsets of specific cardinality 2).
Thus $\emptyset \in P_{X,n,k}$.
Define $h(x) = 1$ for $x \in P_{X,n,k}$. 
$Y = \emptyset$ for all finite sets $Y$ such that $|Y|=0$.
$|Y|=0$ for all $Y \in P_{X,n,k}$.
$h$ is injective.
$h(\emptyset) = 1$.
$h$ is a surjection from $P_{X,n,k}$ onto $\Seq{1}{1}$.
$h$ is a bijection between $P_{X,n,k}$ and $\Seq{1}{1}$.
Hence $|P_{X,n,k}| = 1$ and $|P_{n,k}| = 1$. End.


Case $k>0$.

$\Seq{1}{k} \subseteq \Seq{1}{n}$.
$|\Seq{1}{k}|=k$.\\
$\Seq{1}{k} \in P_{\Seq{1}{n},n,k}$.
$P_{\Seq{1}{n},n,k}$ is nonempty.
Take a bijection $f$ between $\Seq{1}{n}$ and $X$.
$f[\Seq{1}{k}] \subseteq X$.

(1) For all sets $a,b$ and all maps $g$ from $a$ to $b$ if $b = g[a]$ then $g$ is a surjection from $a$ onto $b$.

$x \subseteq \Seq{1}{n}$ for every $x \in P_{\Seq{1}{n},n,k}$.

Define $F(x)=f[x]$ for $x \in P_{\Seq{1}{n},n,k}$.

$f \upharpoonright x$ is injective for all $x \in P_{\Seq{1}{n},n,k}$.

Let us show that $f \upharpoonright x$ is a surjection from $x$ onto $f[x]$ for all $x \in P_{\Seq{1}{n},n,k}$ (by 1).
Let $x \in P_{\Seq{1}{n},n,k}$.
Let $g=(f \upharpoonright x)$.
$\dom(f \upharpoonright x) = \dom(f) \cap x$.
$x \subseteq \Seq{1}{n}$.
$x \subseteq \dom(f)$.
$\dom(f) \cap x = x$.
$\dom(f \upharpoonright x) = x$.
$\dom(f \upharpoonright x) = \dom(g)$.

Let us show that $f[x] = g[x]$.
$f(x') \in f[x]$ for all $x' \in x$.
$((f \upharpoonright x)[x]) \subseteq f[x]$.
$g[x] \subseteq f[x]$.
Qed.

$g$ is a surjection from $x$ onto $g[x]$ (by 1).
Qed.

$f \upharpoonright x$ is a bijection between $x$ and $f[x]$ for all $x \in P_{\Seq{1}{n},n,k}$.

Let us show that ($f[x]$ is finite and $|f[x]|=k$) for all $x \in P_{\Seq{1}{n},n,k}$.
$f \upharpoonright x$ is a bijection between $x$ and $f[x]$ for all $x \in P_{\Seq{1}{n},n,k}$.
Let $x \in P_{\Seq{1}{n},n,k}$.
Let $g=(f \upharpoonright x)$.\\
(2) $g$ is a bijection between $x$ and $g[x]$.\\
$x \subseteq \Seq{1}{n}$.
$x$ is finite.
$g[x]$ is finite.
$x \sim g[x]$ (by 2).
Qed.

$F(x)=f[x]$ for all  $x \in P_{\Seq{1}{n},n,k}$.
$F(x)$ is finite for all  $x \in P_{\Seq{1}{n},n,k}$.
$|F(x)| = k$ for all  $x \in P_{\Seq{1}{n},n,k}$.
$f[x] \subseteq X$ for all  $x \in P_{\Seq{1}{n},n,k}$.
$F(x) \subseteq X$ for all $x \in P_{\Seq{1}{n},n,k}$.

Let us show that $F$ is a map from $P_{\Seq{1}{n},n,k}$ to $P_{X,n,k}$.
$\dom(F) = P_{\Seq{1}{n},n,k}$.
Let us show that $F[P_{\Seq{1}{n},n,k}] \subseteq P_{X,n,k}$.
Let $x \in P_{\Seq{1}{n},n,k}$.
Let $x' = F(x)$.
$x' \subseteq X$.
$x'$ is finite.
$|x'|=k$.
$P_{X,n,k} = \{y | y$ is finite set such that $y \subseteq X$ and $|y| = k\}$ (by subsets of specific cardinality 2).
$x' \in P_{X,n,k}$.
Qed.

Qed.

$P_{X,n,k}$ is nonempty.




Let us show that $F$ is a surjection from $P_{\Seq{1}{n},n,k}$ onto $P_{X,n,k}$.

Let $x \in P_{X,n,k}$.
$P_{X,n,k} = \{y | y$ is finite set such that $y \subseteq X$ and $|y| = k\}$ (by subsets of specific cardinality 2).
$x \subseteq X$.
$f$ is a bijection between $\Seq{1}{n}$ and $X$.

Then $\inv{f}$ is a bijection between $X$ and $\Seq{1}{n}$.
$\inv{f} \upharpoonright x$ is injective.
$(\inv{f} \upharpoonright x)[x]  \subseteq (\inv{f} \upharpoonright x)[x]$.
$\inv{f} \upharpoonright x$ is a surjection from $x$ onto $(\inv{f} \upharpoonright x)[x]$.
$\inv{f}[x] = (\inv{f} \upharpoonright x)[x]$.
$\inv{f} \upharpoonright x$ is bijection between $x$ and $\inv{f}[x]$.
$|x| = k$.

Let us show that $|(\inv{f}[x])| = |x|$.
For all finite sets $a,b$ if $a$ and $b$ are equinumerous then $|a|=|b|$.
Therefore $\inv{f}[x] \subseteq \Seq{1}{n}$.
$\inv{f}[x]$ is a finite set.
$x$ is a finite set.
$((\inv{f} \upharpoonright x)[x]) = \inv{f}[x]$.
$((\inv{f} \upharpoonright x)[x])$ is finite.
$(\inv{f}[x]) \sim x$.
$(\inv{f}[x])$ and $x$ are equinumerous.
Qed.

Therefore $|(\inv{f}[x])|=k$.
$\inv{f}[x]$ is a finite set.
$\inv{f}[x] \subseteq \Seq{1}{n}$.
$\Seq{1}{n}$ is a finite set such that $|\Seq{1}{n}|=n$.
$P_{\Seq{1}{n},n,k} = \{y|y$ is finite set such that $y \subseteq \Seq{1}{n}$ and $|y| = k\}$(by subsets of specific cardinality 2).
$(\inv{f}[x]) \in P_{\Seq{1}{n},n,k}$.
$f[(\inv{f}[x])] = x$.
$F(\inv{f}[x]) = x$.

Qed.

Let us show that $F$ is injective.
$P_{\Seq{1}{n},n,k}$ is nonempty.
Let $x,y \in P_{\Seq{1}{n},n,k}$.
Assume $F(x) = F(y)$.

Let us show that $x = y$.
$f[x] = f[y]$.
$f$ is a bijection between $\Seq{1}{n}$ and $X$.
$\inv{f}$  is a bijection between $X$ and $\Seq{1}{n}$.
$f$ is injective.
$(\inv{f}[f[x]]) = (\inv{f}[f[y]])$.
For all maps $h$ and all $a\in\dom(h)$ if $h$ is a bijection between $\dom(h)$ and $\range(h)$ then $\inv{h}(h(a))=a$.

Let us show that $\inv{f}[f[y]]=y$.
Let $y' \in y$.
For all maps $h$ and all $a\in\dom(h)$ if $h$ is a bijection between $\dom(h)$ and $\range(h)$ then $\inv{h}(h(a))=a$.
$f$ is a bijection between $\dom(f)$ and $\range(f)$.
$y' \in \dom(f)$.
$\inv{f}(f(y')) = y'$.
Qed.

Let us show that $\inv{f}[f[x]]=x$.
Let $x' \in x$.
For all maps $h$ and all $a\in\dom(h)$ if $h$ is a bijection between $\dom(h)$ and $\range(h)$ then $\inv{h}(h(a))=a$.
$f$ is a bijection between $\dom(f)$ and $\range(f)$.
$x' \in \dom(f)$.
$\inv{f}(f(x')) = x'$.
Qed.

Qed.
Qed.

$F$ is a bijection between $P_{\Seq{1}{n},n,k}$ and $P_{X,n,k}$.

Let us show that $P_{X,n,k}$ and $P_{\Seq{1}{n},n,k}$ are finite and $|P_{X,n,k}|=|P_{\Seq{1}{n},n,k}|$.
For all finite sets $a,b$ if $a$ and $b$ are equinumerous then $|a|=|b|$.
$P_{X,n,k} \subseteq \pow(X)$.
$P_{X,n,k}$ is a finite set.
$P_{\Seq{1}{n},n,k}$ is a finite set.
$P_{\Seq{1}{n},n,k} \sim P_{X,n,k}$.
$P_{\Seq{1}{n},n,k}$ and $P_{X,n,k}$ are equinumerous.
Qed.

Qed.
Qed.
\end{proof}

\begin{theorem}
Let $n,k$ be natural number.
Let $X$ be a finite set such that $|X|=n$.
Then $P_{X,n,k}$ is finite and $|P_{X,n,k}|=\binom{n}{k}$.
\end{theorem}
\begin{proof}
Case $k \leq n$.
$|P_{X,n,k}|=|P_{n,k}|$.
End.

Case not $k \leq n$.
Thus $k > n$.
Then $P_{X,n,k} = \emptyset$.
$\binom{n}{k}=0$.
Therefore $|P_{X,n,k}| = |\emptyset| = 0 = \binom{n}{k}$.
End.
\end{proof}


\end{forthel}

\end{document}
